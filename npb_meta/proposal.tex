\documentclass[8pt]{article}
\usepackage{titling}
% \usepackage{nips10submit_e,times}
% setfont command is only available after usepackage{fontspec}
%\usepackage{ifxetex}
%\ifxetex
\usepackage{fontspec}
\usepackage{authblk}
%\setmainfont{Minion Pro}
\setmainfont{TeX Gyre Pagella}
\usepackage[dvipsnames,usenames]{xcolor}
% \usepackage[opticals,medfamily,minionint,footnotefigures]{MinionPro}


\usepackage[lotdepth, lofdepth]{subfig}
\usepackage{ucs}
\usepackage[utf8x]{inputenc}
% \usepackage{times}
\usepackage{supertabular}

%% Set up bold MinionPro math
\usepackage{bm}
\usepackage{natbib}
\setlength{\bibsep}{0.3pt}

%%% Want to change the section head of the bib??
% \AtBeginDocument{\renewcommand\LITERATURE CITED}}

%%% Document layout, margins
\usepackage{geometry}
\geometry{letterpaper, textwidth=6.5in, textheight=8in, marginparsep=1em}
\setlength{\parskip}{-.25cm plus2mm minus2mm}
\setlength{\droptitle}{-7.5em} 
%%% Document layout, margins
\usepackage{geometry}
\geometry{letterpaper, textwidth=6.5in, textheight=8in, marginparsep=1em}
\usepackage{amsmath, epsfig}
\usepackage{amsfonts}
% \usepackage{amssymb}
\usepackage{algorithm}
\usepackage{algorithmic}
% \usepackage{easybmat}
\usepackage{footmisc}
\usepackage{fullpage}
\renewcommand\algorithmiccomment[1]{// \textit{#1}}
\newtheorem{theorem}{Theorem}[section]
\newtheorem{lemma}[theorem]{Lemma}
\newtheorem{proposition}[theorem]{Proposition}
\newtheorem{corollary}[theorem]{Corollary}
\newtheorem{assumption}[theorem]{Assumption}
\newcommand\independent{\protect\mathpalette{\protect\independenT}{\perp}}
\def\independenT#1#2{\mathrel{\rlap{$#1#2$}\mkern2mu{#1#2}}}
%
\title{Healthcare Data Meta-Analysis via Gaussian Processes}

\author[1]{Wei Wang}
\vspace{-4ex}
\affil[1]{\small{Department of Statistics, Columbia University}}
\begin{document}
\pagenumbering{gobble}
\date{}
\maketitle

\vspace{-6ex}
% \begin{abstract}
% \end{abstract}

\section{Introduction}
Meta-analysis aims to synthesize results from multiple studies and identify
patterns and heterogeneity in relation to the contexts of different
studies. Open Health Data, which promote data sharing and research synthesis
from often heterogeneous sources, pose challenges for traditional
meta-analytical framework, both in terms of sheer scale and modeling
strategies. Bayesian Non-Parametric methods, in particular Gaussian Processes,
hold major promise in addressing these issues.

\section{Literature Review}
Due to the constraints of data collection, traditional meta-analysis focus
mostly on synthesizing study-level summary results \citep{dersimonian1986meta}.
Deidentified Individual-Patient Data (IPD) offers detailed demographic and
medical covariates that can help disentangle incongruities constantly found in
meta-analysis. Based on parametric Bayesian Hierarchical Modeling, there has
developed a body of literature on IPD meta-analysis, particularly in clinical
trial settings \citep{higgins2001meta,
  tudorsmith2005investigating}. % However, a
% lack of solid causal framework, such as the standard potential outcome model
% , often renders the meta-analysis results causally
Recently, \citep{sobel2014meta} proposes an extension of the standard potential
outcome framework \citep{rubin1978bayesian} for causal inference, and applying
the framework to the analysis of Vioxx clinical trials.  On the other hand,
recent developments in Bayesian non-parametric methods
\citep{hjort2010bayesian}, in particular Gaussian process regression
\citep{williams2006gaussian}, demonstrate the possibility of flexible modeling
of large scale structured data sets. Combining a solid causal framework with
hierarchical Gaussian process models in healthcare meta-analysis is my main
motivation.

\section{Proposed Work}
The most salient feature of meta-analysis is the hierarchical
structure. Individual studies form the natural atomic blocks, on top of which a
myriad of structure induced by factors such as indications studied, protocols
followed and population targeted could exists. Hierarchical Gaussian processes
models are effective in flexibly modeling the mean functions as well as
correlation structures \citep{ikemoto2009generalizing, williams2009multi}. This
could be exploited to enable borrowing strength in meta-analysis; for example,
studies in which certain demographics are abundant can supplement studies that
lack those demographics, thus providing a higher resolution picture of the
patterns and heterogeneity as well as the driving factors. Computationally,
since time complexity of Gaussian processes is cubic of the sample size, various
approximation schemes have been proposed for different application areas
\citep{archambeau2007gaussian, banerjee2008gaussian}. I would like to evaluate
the suitability and effectiveness of different approximation schemes in the
context of healthcare data meta-analysis.

\bibliographystyle{plain}
\small{\bibliography{ref}}


\end{document}
