\chapter{Causal Inference of Meta-Analysis via Gaussian Processess}

Meta-Analysis, the synthesis of evidence from multiple study sources,
has become increasing popular in fields such as education, psychology
and public health (Cooper, Hedges, and Valentine 2009). The major
obstacle for meta-analysis is the interpretation and proper handling of
study-by-study heterogeneity, e.g., the estimated treatment effect in
study 1 is different from in study 2. Traditional approaches tend to
focus on developing novel ways of weighting different estimates based on
certain measure of study-level uncertainty and/or quality. (Sobel,
Madigan, and Wang 2016), however, approaches this problem from a formal
causal inference perspective, proposing an extended potential outcome
framework for meta-analysis. Although it is not a panacea for explaining
and accounting for heterogeneities, this approach indeed clarifies the
originations of heterogeneities and help researchers to think more
clearly on underlying assumptions that are often overlooked. In this
chapter, I review briefly the extended potential outcome framework
discussed in (Sobel, Madigan, and Wang 2016). However, (Sobel, Madigan,
and Wang 2016) use simple linear models for analysis. In remaining part
of this thesis, I develop a non-parametric model that explicitly handles
heterogeneities across studies based on Gaussian Processes (GP). As it
is well known (Williams and Rasmussen 2006), GP allows for flexible
modeling of response functions and admits fully probabilistic inference.
Finally, a real educational intervention data set is analyzed with this
model to illustrate. This chapter is joint work with Michael Sobel and
David Madigan, and part of it is published in (Sobel, Madigan, and Wang
2016).

\section{Meta-Analysis}\label{meta-analysis}

Meta-analyses combine data from distinct but related studies for higher
resolution inference and more nuanced understanding of the effect under
investigation. Originally hailed in medical and education research,
meta-analyses are gaining traction as the awareness for open data is
increasing across all scientific communities.

However, traditional meta-analyses are mostly conducted based on
extracting and combining study-level effect summaries, since access to
individual-participant level data tend to be inherently difficult to
obtain. In this framework, researchers extract effect size estimates
\(y_s\) and standard errors \(\sigma_s^2\), where study index
\(s\in \{1,\ldots,S\}\) and \(S\) is the total number of studies. To
handle effect size heterogeneity, typically a random effect model is
used (DerSimonian and Laird 1986), in which all study effect sizes are
assumed to be a random sample of a underlying hyper-population of effect
sizes, i.e.

\begin{align*}
    y_s&=\mu_s + \sigma_s^2\\
    \mu_s &\sim \mathcal{N}(\mu_0, \tau^2)
\end{align*}

\noindent Admittedly, meta-analysis based on study-level summary is
still effective when the effects are homogeneous and different studies
sample from similar populations; nevertheless, they are prone to
well-known statistical fallacies, such as ecological bias, when the
underlying populations and effects are heterogeneous, as it is often the
case in real data.

\subsection{Individual-Participant
Meta-Analyses}\label{individual-participant-meta-analyses}

Individual-Participant Data (IPD) Meta-Analyses are becoming more and
more common, thanks to the increasing availability of original data
(Higgins et al. 2001). It has been argued that IPD data increases the
power of analysis (Cooper and Patall 2009) and more robust to
heterogeneous effects sizes and populations. To account for between
study heterogeneity in treatment effects, the use of covariates and/or
random effects models is often recommended (Aitkin 1999; Tudur Smith,
Williamson, and Marson 2005). The random effects models can be seen as
Bayesian hierarchical models (Gelman and Hill 2006), based on the
justification that conditioned on an appropriate set of covariates, both
individual-level and study-level, the residual heterogeneities are
exchangeable. There are mature softwares for fitting various types of
Bayesian hierarchical models, including Generalized Linear Models and
Proportional Hazard Models (Bates et al. 2015; ``Stan: A c++ library for
probability and sampling, version 2.8.0'' 2015; Therneau 2012).

Despite its convenient form and ease of inference, traditional IPD
meta-analysis based on parametric hierarchical models suffer from two
problems. The first is the lack of formal causal framework. It is
difficult to pinpoint the causal interpretation of the effect estimates
from a traditional IPD hierarchical model. Consider the following
example, education researchers try to determine the effect of a new
intervention program, applied to different classroom and administered by
different teachers. In this case, the heterogeneity might come from
either the different teachers or the different populations of schools,
or both. It is often unclear whether the effect estimate based on
traditional methods are averages over the teachers, or over the schools,
or both. The second problem is the inflexible form of the parametric
model. Traditional parametric model requires explicit modeling
assumptions from the researchers, which makes the model sensitive to
model specifications and facilitate potential cheery-picking.
Non-parametric modeling allows flexible functional form and requires
little manual tuning from the researchers.

\section{A Potential Outcome Framework for
Meta-Analysis}\label{a-potential-outcome-framework-for-meta-analysis}

(Sobel, Madigan, and Wang 2016) put meta-analysis on a concrete causal
foundation by introducing an extended potential outcome framework. I
will discuss the key ideas in this framework.

\subsection{Potential Outcomes}\label{potential-outcomes}

Potential Outcomes Framework (Rubin 2011) defines causal effects as
comparisons of outcomes under hypothetical counter-factual treatment
assignment. For example, with binary treatment \(Z\in \{0,1\}\), the
causal effect of treatment \(Z\) on individual \(i\) can be defined as
\(y_i(1)-y_i(0)\). Typically, researchers are interested in estimating
quantities such as the population average treatment effect (PATE)

\begin{equation*} E(Y(1)-Y(0)), \end{equation*}

\noindent and the population average treatment effect on the treated
(PATT)

\begin{equation*}
    E(Y(1)-Y(0)\mid Z=1)
\end{equation*}

\noindent The key assumption in causal inference is the ignorability
assumption (or unconfoundedness assumption) (Rosenbaum and Rubin 1983),
which states that given a set of observed covariates, the treatment
assignment \(Z\) is independent of the potential outcomes
\((Y(0), Y(1))\)

\begin{equation*}
    Y(0),Y(1) \perp Z \mid X
\end{equation*}

\noindent In the case of randomized experiment, this assumption is
trivially met without any covariates \(X\). Under ignorability
assumption,

\begin{align*}
     &E(Y|X, z=1)-E(Y|X, z=0)\\
    =&E(Y(1)|X, z=1)-E(Y(0)|X, z=0)\\
    =&E(Y(1)|X)-E(Y(0)|X)\\
    =&E(Y(1)-Y(0)|X)
\end{align*}

\noindent and thus causal effect can be identified from observations.

\subsection{Extended Potential
Outcomes}\label{extended-potential-outcomes}

In the case of meta-analysis, consisting \(S\) studies and \(Z\)
treatment arms, the potential outcomes \(\bm Y\) for individual \(i\)
can be defined as a matrix

\begin{equation*}
\bm Y_i=
\begin{pmatrix}
y_i(1,1) & y_i(1,2) & \cdots & y_i(1,Z)\\
y_i(2,1) & y_i(2,2) & \cdots & y_i(2,Z)\\
\vdots & \vdots & \ddots & \vdots\\
y_i(S,1) & y_i(S,2) & \cdots & y_i(S,Z)
\end{pmatrix}
\end{equation*}

\noindent With this notation, some commonly discussed meta-analytical
estimates can be interpreted in a causal way. For example, assuming
there are only two level of treatment (0 and 1) and the causal
comparison is the difference, \emph{study-specific} treatment effect for
study \(s\) is \(E(y(z,s)-y(z^\prime, s))\). Note that this is different
from \emph{study-level} treatment effect \(\theta_s\) in random effects
models, which is \(E(y(z,s)-y(z^\prime, s)\mid S=s)\). Below I will
discuss conditions that will connect these two quantities.

In the context of meta-analyses, unconfoundedness can be recast as
unconfoundedness within each study, i.e.,

\begin{equation*}
    Y(0, s),Y(1, s) \perp Z \mid X, S=s
\end{equation*}

However, this assumption is not sufficient for identifying causal
effects in meta-analysis. One added layer for complexity of
meta-analysis is the confounding of study selection. Consider an example
of clinical trials. If some studies sample from mostly young patients
while some other studies sample from mostly elderly patients, and the
treatment is more effective on younger patients, then heterogeneities in
treatment effects across studies would arise. Hierarchal models without
adequately addressing this selection problems would result in misleading
results.

However, study selection is not the only factor contributing to
heterogeneities in treatment effects across studies. One lingering
question is whether the same treatment \(z\) is implemented identically
in all studies, or in another word, whether
\(Y_i(s_1, z)\disteq Y_i(s_2, z)\) for all pair of
\(s_1, s_2 \in \{1,\dots, S\}\), where \(\disteq\) stands for equal in
distribution. Consider an example of education intervention, in which
interventions are carried out by teachers with various experience
levels, then it is reasonable to question whether
\(Y_i(s_1, z)\disteq Y_i(s_2, z)\) holds.

Two assumptions from (Sobel, Madigan, and Wang 2016) codify these two
sources of heterogeneities.

A1. \emph{Weak response consistency assumption for treatment \(z\)}: For
any \(z\in\{1,\dots,Z\}\) and any pair \(s_1, s_2\in\{1,\ldots,S\}\),

\begin{equation*}
Y_i(s_1, z)\disteq Y_i(s_2, z)
\end{equation*}

A2. \emph{Unconfounded study selection}:

\begin{equation*}\bm Y=
\begin{pmatrix}
y(1,1) & y(1,2) & \cdots & y(1,Z)\\
y(2,1) & y(2,2) & \cdots & y(2,Z)\\
\vdots & \vdots & \ddots & \vdots\\
y(S,1) & y(S,2) & \cdots & y(S,Z)
\end{pmatrix}
 \perp S \mid X\end{equation*}

.

From a modeling perspective, these two assumptions cannot be
distinguished from one other. Thus (Sobel, Madigan, and Wang 2016)
suggests that researchers first assess the plausibility of the two
assumptions based on the characteristics of the studies, and typically
assume one of these two to hold and then build models to see whether the
heterogeneity could be accounted for by the other assumption. From a
Bayesian point of view, I can use a very general model, and encode
regularization through appropriate prior distributions to allow for
reasonable separation of these two sources of heterogeneities. This will
be the topic of the following sections.

\section{Meta-Analysis using Bayesian
Non-parametrics}\label{meta-analysis-using-bayesian-non-parametrics}

Traditionally, causal inference using potential outcomes focuses on two
questions. Modeling of the treatment assignment process \(p(z\mid x)\),
also known as the propensity score, and modeling of the scientific
process of how responses relate to treatment and covariates
\(p(y\mid z, x)\), also known as the response surface (Rubin 2005). A
myriad of methods based on the either treatment assignment mechanism
(e.g., propensity score matching), or response surface modeling (e.g.,
regression), or combination of these two (e.g., the doubly-robust
method), has been proposed for causal inference of observational data.

Recently, following the advances in Bayesian non-parametric models ,
(Hill 2011) proposed a model that focuses on accurately estimating the
response surface using flexible Bayesian Additive Regression Trees, or
BART (Chipman, George, and McCulloch 2010). Besides the well-known
benefits of being robust to model misspecifications and being able to
capture highly non-linear and interaction patterns, Bayesian
non-parametric models provide natural and coherent posterior intervals
to convey inferential uncertainty.

\subsection{Gaussian Processes}\label{gaussian-processes}

Gaussian Processes (GP) have become a popular tool for nonparametric
regression. A random function \(f:\mathcal{X}\rightarrow\mathbb{R}\) is
said to follow a GP process with kernel \(k\) if any finite-dimensional
marginal of it is Gaussianly distributed, i.e.

\begin{equation*}
    f(\bm x) \sim \mathcal{N}(\bm\mu, \bm K_{\bm x, \bm x}), \forall\ \bm x \in
    \mathbb{R}^{d} \text{ and } d
\end{equation*}

where \(\bm K_{\bm x, \bm x}\) is the Gram matrix of kernel \(k\). The
key component in a GP model is the kernel \(k\), a semi-definite
function defined on \(\mathcal{X}\times\mathcal{X}\) that encodes the
structure. Judiciously choosing \(K\) is the most important part of
fitting a GP model.

A large part of its popularity is probably due to the fact it can be
interpreted as a generalization of linear regression with Gaussian
errors, the predominant model for parametric regression. In fact,
according to Mercer's Theorem (Williams and Rasmussen 2006), kernel
\(k\) can be decomposed into

\begin{equation*}
k(x, x^\prime)=\sum_{i=1}^\infty\lambda_i\phi_i(x)\phi^\intercal_i(x^\prime)
\end{equation*}

\noindent where \(\lambda_i\) and \(\phi_i\) are respective eigenvalues
and eigenfunctions of kernel \(k\) with respect to a measure \(\mu\),
i.e.,

\begin{equation*}
\int k(x, x^\prime)\phi_i(x) \,d\mu(x)=\lambda\phi_i(x^\prime),
\end{equation*}

\noindent Then GP can be considered as a basis expansion method that
maps input \(x\) to an infinite dimensional space via the infinite
series of functions \(\{\phi_i(x)\}_{i=1}^\infty\).

\subsection{Inference for Standard GP}\label{inference-for-standard-gp}

Standard GP model for \(N\) observation pairs \((y_i, \bm x_i)_{i=1}^N\)
is

\begin{align*}
    y_i\mid f &\sim \mathcal{N}(f(\bm x_i), \sigma^2) \\
    f &\sim GP(0, k)
\end{align*}

For a given kernel \(k\), the marginal distribution of \(\bm y\) is

\begin{equation*}
\bm y \sim \mathcal{N}(0, K_{\bm x, \bm x}+\sigma^2I_N)
\end{equation*}

\noindent where \(K_{\bm x, \bm x}\) is the Gram matrix of kernel \(k\)
whose entries are \(k(x_i, x_j)\). The predictive distribution at new
points \(\bm X^\star\) is

\begin{align*}
\bm y^\star \mid \bm X^\star, \bm y, \bm X \sim \mathcal{N}(&K_{\bm X^\star, \bm X}(K_{\bm X,
\bm X}+\sigma^2I_N)^{-1}\bm y, \\
& K_{\bm X^\star, \bm X^\star}-K_{\bm X^\star, \bm X}(K_{\bm X, \bm X}+\sigma^2I_N)^{-1}K_{\bm X^\star, \bm X}^\intercal)
\end{align*}

\noindent For inference on hyperparameters, e.g., parameters governing
the kernels, a standard practice is to maximize log marginal likelihood

\begin{align*}
\log p(\bm y \mid \bm X, \theta)&=\log \int p(\bm y \mid f, \bm X, \theta) p(f)\, df \\
    & \propto
    -\big[\bm y^\intercal(K_{\bm X, \bm X}(\theta)+\sigma^2I_N)^{-1} \bm y + \log\text{det}(K_{\bm X, \bm X}(\theta)+\sigma^2I_N)\big]
\end{align*}

\noindent and plug in the MAP (maximum a posteriori) \(\hat \theta\)
into the predictive distribution of new points \(\bm X^\star\).

As it is well known, despite the simplicity of the procedure for GP
inference, the main difficulty lies in the matrix inversions required
for both estimating hyperparameters and predicting at new points, which
involves \(\mathcal{O}(N^3)\) time complexity with \(N\) being the
number of observations. Data sets that have more than several thousands
observations are already prohibitively expensive for computation. In
those cases, a number of approximation methods such as low-rank
approximations of the Gram kernel matrix, known as Nystr\{"o\}m method
(Williams and Seeger 2001), and judicious selections of subset of
observations (Banerjee et al. 2008) are often recommended.

\subsection{GP with Hierarchical
Structure}\label{gp-with-hierarchical-structure}

GP can be extended to handle group structure. A common approach from
machine learning perspective is to pose this question as a multi-task
learning problem (Bonilla, Chai, and Williams 2007; Yu, Tresp, and
Schwaighofer 2005), in which the objective function's values are
vectors, or even matrices. In fact, in the case of the potential outcome
framework of meta-analysis outlined above, the outcomes are
study-by-treatment matrices. In this setting, the curse of
dimensionalities become even more acute, as the sample size effectively
multiples by the number of outcomes under investigation. One remedy is
to place restriction on the structure of kernels. For example, assuming
the kernel is separable, the finite dimensional marginals of the
vector-valued random function \(\bm f\) is a matrix normal distributed

\begin{equation*}
\text{vec}{\bm f(\bm x)} \sim \mathcal{N}(\bm \mu, \bm K_{\bm x, \bm x} \otimes \bm U_{\text{task}})
\end{equation*}

\noindent Assuming separability can significantly reduce the
dimensionality of the problem, and properties of Kronecker product can
be used to make the inference efficient (Gilboa, Saatçi, and Cunningham
2015). However, this approach works best for the case of ``complete
design'', meaning every combination of predictors values have been
observed. This assumption is reasonable for areas such as computer
experiments and robotics, where experiments can be artificially planned
at pre-specified values of X's; but it is virtually impossible in fields
such as education and public health, where researchers' control over
data collection is very limited.

\subsubsection{Incorporating Group Structure into
Kernels}\label{incorporating-group-structure-into-kernels}

However, it is relatively straightforward to directly code the group
structure into kernels. Take the most popular kernel, square
exponential, for example, discrete group indicator terms can be added by
using delta metrics, i.e., 0 if two observations are from the same
group, and 1 otherwise. Further, it is often important to add
group-level predictors in hierarchical models (Gelman and Hill 2006), as
it can account for the variations that cannot be explained by
categorical group membership.

\begin{equation*}
\kappa(x_i, x_j)=\sigma^2\exp{-\frac{1}{2}\sum_{k=1}^d \frac{(x_{ik}-x_{jk})^2}{l_k^2}}
\end{equation*}

\noindent In square exponential kernels, the length scale \(l_k\)
governs the correlation scale in input dimension \(k\) and the magnitude
\(\sigma^2\) controls the overall variability of the process. Thus the
magnitudes of the length scale \(l_k\) can be used for feature
selection; the larger the length scale, the more important the
corresponding feature.

\subsection{Network Meta-Analysis}\label{network-meta-analysis}

The compact form of Kronecker product assumes a block-design structure,
i.e., the same set of \(x\)'s are observed for every combination of
study and treatment. In reality, of course, this is not the case; causal
inference, in particular, is about filling those holes, i.e., potential
outcomes, in hypothetical combination of study and treatment. In fact,
researchers only observe a small fraction of the array of matrices
\(\{\bm y_i\}_{i=1}^\infty\), namely \(\frac{1}{ST}\) of all potential
outcomes. However, this framework is general enough to handle a lot of
particular problems.

Network Meta-analysis (Lumley 2002) deal with treatment pair comparisons
that depend on indirect evidence. For example, if treatment A and
treatment B are not assigned in any of the studies at the same time, and
thus researchers have to resort to indirect comparisons, e.g., treatment
C co-occurs with treatment A and treatment B in some of the studies.
Since traditional analysis tend to handles one comparison at a time,
violations of natural constraints are frequent, e.g., AB+BC\(\neq\)BC.
More sophisticated models are proposed to handle this so-called
``inconsistency'' (Higgins et al. 2012), which makes the models
unnecessarily complex and is detrimental to intuitive understanding. The
framework outlined in this chapter, however, naturally deal with network
meta-analysis, since it considers all possible treatment at the same
time and thus have those constraints built in organically.

\section{Real Data Example}\label{real-data-example}

The demonstrative example I use is the STAR (Student-Teacher Achievement
Ratio) project. It was approved by Tennessee state legislature and began
in 1985 to study the effect of early grades class size on student
achievement in Tennessee. The study is a state-wide randomized
experiment applied to over 7,000 pupils from 79 schools and last for 4
years. Each student was randomly assigned to one of three class types,
class of 13 to 15 students, class of 22 to 25 students, and class of 22
to 25 students with a paid teaching aid. Outcomes of end-of-year test
scores were used to assess the performance of those students in areas of
math, reading and study skills. Classroom teachers were also randomly
assigned to the classes they would teach. The interventions were
initiated as the students entered school in kindergarten and continued
through third grade, based on the common belief that early intervention
has persistent effects well into later lives of the students. Due to its
size and ambition, STAR is perhaps the most important education study in
history. Numerous studies have been devoted to analyzing the STAR data,
on both immediate effects, e.g., test scores at the end of the year of
intervention (krueger1997experimental; Hanushek, Mayer, and Peterson
1999; Word and others 1990), and persistent effects, e.g., test scores
several years after the intervention or even earning as an adult (Chetty
et al. 2010).

The public access data set is collected from Project STAR Web site at
\url{http://www.heros-inc.org/star.htm}, with information on the student
demographics, test scores, treatment assignments over the intervention
years, information of the teachers, and school situations et al. Due to
its richness, STAR project data can be investigated in many different
facets. For the sake of simplicity, I look at just one outcome, scaled
math test score, in one intervention year, the 1st grade. Thus I can
focus on the meta-analytic part of the data, without being distracted by
the longitudinal aspect of the data, which is a nuisance for the
discussion. To be precise, I only study the students who participated
STAR project in their first grades and use their scaled math score at
the end of the first year as outcomes. Further, as mentioned previously,
data sets with thousands of observations pose prohibitive computational
burden on GP. So I select a subset of the data set, including only 8
biggest schools for the first grade. The sample size of this restricted
data set is about 1,000.

Characteristics such as student gender, ethnicity, receiving free lunch
or not are included in the data set; furthermore, I can determine the
general neighborhood economic situations by calculating the proportion
of students receiving free lunch, a school-level predictor. As for types
of treatment, as I mentioned, there are three types of treatment,
small-size classroom, regular-size classroom and regular with a paid
teaching aid. Instead of combining regular and regular with an aid, an
approach adopted by most of previous literature, I treat them as
separate interventions. All of the above predictors are fed into a
square exponential kernel, with a delta metric for discrete predictors
including type of treatment and school ID. Computations are conducted
via an MATLAB toolbox \texttt{GPStuff} (Vanhatalo et al. 2013).

While traditional parametric inference focuses on interpretation of some
model coefficient estimates, whose validity greatly hinges on the
validity of the model specifications, non-parametric inference attempts
to construct the response surfaces and yields much more faithful
uncertainty estimates when extrapolating. The results of inference are
presented as a series of figures below. In each of the figures,
predictive estimates for a student in a certain demographic subgroup,
e.g., a minority female pupil receiving free lunch, were she in each of
the 8 schools, are presented side-by-side under three different
treatments. The schools are arranged in order of proportions of students
receiving free lunch, a proxy of neighborhood economic situation, from
most affluent to the left to the most deprived to the right. Figure 1
denotes a minority pupil with free lunch, figure 2 a minority pupil with
paid lunch, figure 3 a white pupil with free lunch and figure 4 a white
pupil with paid lunch.

\begin{sidewaysfigure}[p!]
  \includegraphics[width=\textwidth]{"f1"}
  \caption{Above: Counterfactual scaled math scores with one standard deviation if a minority female pupil receiving free lunch were assigned to 8 different schools and 3 different treatments. Below: Counterfactual scaled math scores with one standard deviation if a minority male pupil receiving free lunch were assigned to 8 different schools and 3 different treatments. Schools are ordered from left to right by the proportions of student receiving free lunch.}
  \label{fig:f1}
  \end{sidewaysfigure}

\begin{sidewaysfigure}[p!]
\includegraphics[width=\textwidth]{"f2"}
  \caption{Above: Counterfactual scaled math scores with one standard deviation if a minority female pupil not receiving free lunch were assigned to 8 different schools and 3 different treatments. Below: Counterfactual scaled math scores with one standard deviation if a minority male pupil not receiving free lunch were assigned to 8 different schools and 3 different treatments. Schools are ordered from left to right by the proportions of student receiving free lunch.}
  \label{fig:f2}
\end{sidewaysfigure}

\begin{sidewaysfigure}[p!]
  \includegraphics[width=\textwidth]{"f3"}
    \caption{Above: Counterfactual scaled math scores with one standard deviation if a white female pupil receiving free lunch were assigned to 8 different schools and 3 different treatments. Below: Counterfactual scaled math scores with one standard deviation if a white male pupil receiving free lunch were assigned to 8 different schools and 3 different treatments. Schools are ordered from left to right by the proportions of student receiving free lunch.}

  \label{fig:p3}
\end{sidewaysfigure}

\begin{sidewaysfigure}[p!]
\includegraphics[width=\textwidth]{"f4"}
    \caption{Above: Counterfactual scaled math scores with one standard deviation if a white female pupil not receiving free lunch were assigned to 8 different schools and 3 different treatments. Below: Counterfactual scaled math scores with one standard deviation if a white male pupil not receiving free lunch were assigned to 8 different schools and 3 different treatments. Schools are ordered from left to right by the proportions of student receiving free lunch.}

  \label{fig:p4}
\end{sidewaysfigure}

There are several interesting points worth noting from the figures.
First, in general, attending smaller class translates to a modest
increase in scaled math test score for the first year pupils, and in
particular, the effects are more pronounced in schools with higher level
of poverty, even after adjusting for proportions of students receiving
free lunch. The heterogeneity of the small class sizes on educational
outcomes have been noted with traditional parametric analysis (Krueger
1997), but most often by adding parametric interaction terms that
requires very specific assumptions; without specifying parametric form,
GP could uncover patterns of heterogeneities. Second, note that
uncertainty is small whenever more observations are available. For
example, in figure 1, which corresponds to poor minority pupils that are
present in schools to the right side of the figures, those schools to
the right indeed carry much shorter error bar as compared to more
affluent schools to the left. Meanwhile, estimates in figure 4 go in the
opposite direction: schools to the left have much shorter error bars,
since white, non-poor pupils are more present in those schools. Although
GP is just doing what it is supposed to do, it is reassuring to know
that fidelity is being preserved for uncertainty estimates. Third, a
sizable amount of variation can be observed for a pupil receiving the
same treatment under different schools. This suggests a departure from
Weak Response Consistency Assumption (A1). This is not surprising,
because other than the same class size, different schools definitely
offer education at different qualities. Take Figure 1 for example, which
focuses on a minority and economically disadvantaged pupil. School 9
tends to do consistently very well across all three treatments, but
school 19 is the school that would benefit the most by adopting a small
class size.

\section{Discussion}\label{discussion}

In this chapter, I discuss an extended potential outcome framework built
for meta-analysis. Comparing with the classic potential outcome
framework, a plethora of counterfactuals with certain structures need to
be created to handle meta-analysis. I then introduce a GP based approach
to tackle this problem. The advantages of GP, and in general any
non-parametric methods, is well-known. In particular, the fidelity of
inferential uncertainty is a very desirable property. The central
question to the extended potential outcome framework is how to
incorporate the group structure. I discuss different ways of building
group structure into Gaussian Processes. The most intuitive and
computationally straight-forward approach is used to re-analyze an
influential educational intervention program, the STAR project on class
size and test scores. Lucid and straight-forward visualization can be
used to display the inferential results, and reveal patterns that are
otherwise hidden in tables of coefficients estimates often seen in
traditional parametric analysis.

GP is a much sought-after field of research recently and reasonably so.
Applying it to casual inference, and in particularly causal inference
with group structure, can surely yield elucidating insights. The scope
of this chapter is quite limited, and I hope it demonstrates the
potential of non-parametric methods in causal inference.

\newpage

\hyperdef{}{references}{\label{references}}
\section*{Bibliography}\label{bibliography}
\addcontentsline{toc}{section}{Bibliography}

\hyperdef{}{ref-aitkin1999meta}{\label{ref-aitkin1999meta}}
Aitkin, Murray. 1999. ``Meta-analysis by random effect modelling in
generalized linear models.'' \emph{Statistics in Medicine} 18(17-18):
2343--2351.

\hyperdef{}{ref-banerjee2008gaussian}{\label{ref-banerjee2008gaussian}}
Banerjee, Sudipto et al. 2008. ``Gaussian predictive process models for
large spatial data sets.'' \emph{Journal of the Royal Statistical
Society: Series B} 70(4): 825--848.

\hyperdef{}{ref-lme4}{\label{ref-lme4}}
Bates, Douglas et al. 2015. ``Fitting linear mixed-effects models using
lme4.'' \emph{Journal of Statistical Software} 67(1): 1--48.

\hyperdef{}{ref-bonilla2007multi}{\label{ref-bonilla2007multi}}
Bonilla, Edwin V, Kian M Chai, and Christopher Williams. 2007.
``Multi-task gaussian process prediction.'' In \emph{Advances in neural
information processing systems}, p. 153--160.

\hyperdef{}{ref-chetty2010does}{\label{ref-chetty2010does}}
Chetty, Raj et al. 2010. \emph{How does your kindergarten classroom
affect your earnings? Evidence from project sTAR}. National Bureau of
Economic Research.

\hyperdef{}{ref-chipman2010bart}{\label{ref-chipman2010bart}}
Chipman, Hugh A, Edward I George, and Robert E McCulloch. 2010. ``BART:
Bayesian additive regression trees.'' \emph{The Annals of Applied
Statistics}: 266--298.

\hyperdef{}{ref-cooper2009relative}{\label{ref-cooper2009relative}}
Cooper, Harris, and Erika A Patall. 2009. ``The relative benefits of
meta-analysis conducted with individual participant data versus
aggregated data.'' \emph{Psychological Methods} 14(2): 165.

\hyperdef{}{ref-cooper2009handbook}{\label{ref-cooper2009handbook}}
Cooper, Harris, Larry V Hedges, and Jeffrey C Valentine. 2009. \emph{The
handbook of research synthesis and meta-analysis}. Russell Sage
Foundation.

\hyperdef{}{ref-dersimonian1986meta}{\label{ref-dersimonian1986meta}}
DerSimonian, Rebecca, and Nan Laird. 1986. ``Meta-analysis in clinical
trials.'' \emph{Controlled Clinical Trials} 7(3): 177--188.

\hyperdef{}{ref-gelman2006data}{\label{ref-gelman2006data}}
Gelman, Andrew, and Jennifer Hill. 2006. \emph{Data analysis using
regression and multilevel/hierarchical models}. Cambridge University
Press.

\hyperdef{}{ref-gilboa2015scaling}{\label{ref-gilboa2015scaling}}
Gilboa, Elad, Yunus Saatçi, and John P Cunningham. 2015. ``Scaling
multidimensional inference for structured gaussian processes.''
\emph{Pattern Analysis and Machine Intelligence, IEEE Transactions on}
37(2): 424--436.

\hyperdef{}{ref-hanushek1999evidence}{\label{ref-hanushek1999evidence}}
Hanushek, Eric A, Susan E Mayer, and Paul Peterson. 1999. ``The evidence
on class size.'' \emph{Earning and Learning: How Schools Matter}:
131--168.

\hyperdef{}{ref-higgins2012consistency}{\label{ref-higgins2012consistency}}
Higgins, JPT et al. 2012. ``Consistency and inconsistency in network
meta-analysis: Concepts and models for multi-arm studies.''
\emph{Research Synthesis Methods} 3(2): 98--110.

\hyperdef{}{ref-higgins2001meta}{\label{ref-higgins2001meta}}
Higgins, Julian et al. 2001. ``Meta-analysis of continuous outcome data
from individual patients.'' \emph{Statistics in Medicine} 20(15):
2219--2241.

\hyperdef{}{ref-hill2011bayesian}{\label{ref-hill2011bayesian}}
Hill, Jennifer L. 2011. ``Bayesian nonparametric modeling for causal
inference.'' \emph{Journal of Computational and Graphical Statistics}
20(1).

\hyperdef{}{ref-krueger1997experimental}{\label{ref-krueger1997experimental}}
Krueger, Alan B. 1997. \emph{Experimental estimates of education
production functions}. National Bureau of Economic Research.

\hyperdef{}{ref-lumley2002network}{\label{ref-lumley2002network}}
Lumley, Thomas. 2002. ``Network meta-analysis for indirect treatment
comparisons.'' \emph{Statistics in Medicine} 21(16): 2313--2324.

\hyperdef{}{ref-rosenbaum1983central}{\label{ref-rosenbaum1983central}}
Rosenbaum, Paul R, and Donald B Rubin. 1983. ``The central role of the
propensity score in observational studies for causal effects.''
\emph{Biometrika} 70(1): 41--55.

\hyperdef{}{ref-rubin2011causal}{\label{ref-rubin2011causal}}
Rubin, Donald B. 2011. ``Causal inference using potential outcomes.''
\emph{Journal of the American Statistical Association}.

\hyperdef{}{ref-rubin2005causal}{\label{ref-rubin2005causal}}
Rubin, Donald B. 2005. ``Causal inference using potential outcomes.''
\emph{Journal of the American Statistical Association} 100(469).

\hyperdef{}{ref-sobel2016}{\label{ref-sobel2016}}
Sobel, Michael E, David B Madigan, and Wei Wang. 2016. ``Meta-analysis:
A causal framework, with application to randomized studies of vioxx.''
\emph{Psychometrika}.

\hyperdef{}{ref-stan-software:2015}{\label{ref-stan-software:2015}}
``Stan: A c++ library for probability and sampling, version 2.8.0.''
2015. \url{http://mc-stan.org/}.

\hyperdef{}{ref-therneau2012coxme}{\label{ref-therneau2012coxme}}
Therneau, Terry. 2012. ``Coxme: Mixed effects cox models.'' \emph{R
package version} 2(3).

\hyperdef{}{ref-tudursmith2005investigating}{\label{ref-tudursmith2005investigating}}
Tudur Smith, Catrin, Paula R Williamson, and Anthony G Marson. 2005.
``Investigating heterogeneity in an individual patient data
meta-analysis of time to event outcomes.'' \emph{Statistics in Medicine}
24(9): 1307--1319.

\hyperdef{}{ref-vanhatalo2013gpstuff}{\label{ref-vanhatalo2013gpstuff}}
Vanhatalo, Jarno et al. 2013. ``GPstuff: Bayesian modeling with gaussian
processes.'' \emph{The Journal of Machine Learning Research} 14(1):
1175--1179.

\hyperdef{}{ref-williams2001using}{\label{ref-williams2001using}}
Williams, Christopher, and Matthias Seeger. 2001. ``Using the nystrom
method to speed up kernel machines.'' In \emph{Proceedings of the 14th
annual conference on neural information processing systems}, p.
682--688.

\hyperdef{}{ref-williams2006gaussian}{\label{ref-williams2006gaussian}}
Williams, CKI, and CE Rasmussen. 2006. \emph{Gaussian processes for
machine learning}. Cambridge: MIT Press.

\hyperdef{}{ref-word1990state}{\label{ref-word1990state}}
Word, Elizabeth R, and others. 1990. ``The state of tennessee's
student/Teacher achievement ratio (sTAR) project: Technical report
(1985-1990).''

\hyperdef{}{ref-yu2005learning}{\label{ref-yu2005learning}}
Yu, Kai, Volker Tresp, and Anton Schwaighofer. 2005. ``Learning gaussian
processes from multiple tasks.'' In \emph{Proceedings of the 22nd
international conference on machine learning}, ACM, p. 1012--1019.
