\documentclass[xetex,mathserif,serif]{beamer}
\usetheme{Frankfurt}
% \usepackage{nips10submit_e,times}
% setfont command is only available after usepackage{fontspec}
% \usepackage{ifxetex}
% \ifxetex
\usepackage{fontspec}
\usepackage{xltxtra}
% \usepackage[sc]{mathpazo}
% \usepackage[BoldFont,CJKnumbb] {xeCJK}
% \defaultfontfeatures{Ligatures={Common}, Mapping={tex-text}}
\setmainfont{Minion Pro}
% \setsansfont{Minion Pro}
% \setmonofont[Scale=0.8]{Monaco}
%   % \setCJKmainfont{Adobe Song Std}
%   % \XeTeXlinebreaklocale "zh"
%   % \XeTeXlinebreakskip = 0pt plus 1pt
% \else

%\usepackage[sc]{mathpazo}
%\usepackage[T1]{fontenc}
% \fi

% converts LaTeX specials (``quotes'' --- dashes etc.) to unicode

% \setromanfont{Liberation Serif} % Roman font is Serif font, \setromanfont has been
% replaced by \setmainfont
% \RequirePackage{mathptmx}
% \setromanfont{FreeSerif}

% can't find free version of Electra LH fonts
% \setromanfont [Mapping=tex-text,Ligatures={Common},BoldFont={ElectraLH-Bold},ItalicFont={ElectraLH-CursiveOsF},BoldItalicFont={ElectraLH-BoldCursiveOsF},SmallCapsFont={ElectraLH-RegularSC}]{ElectraLH-RegularOsF}
% \setsansfont[Mapping=tex-text,BoldFont={Delicious-Bold},ItalicFont={Delicious-Italic},SmallCapsFont={Delicious-SmallCaps}] {Delicious-Roman}


% \usepackage{cmbright}
% \renewcommand\sfdefault{phv}% use helvetica for sans serif
% \renewcommand\familydefault{\sfdefault}% use sans serif by default
% \usepackage[dvipsnames,usenames]{xcolor}
% \usepackage[opticals,medfamily,minionint,footnotefigures]{MinionPro}
\AtBeginSection[] {
  \begin{frame}[plain]
    \frametitle{Overview}
    \tableofcontents[currentsection]
  \end{frame}
  \addtocounter{framenumber}{-1}
}
\graphicspath{{./figures/}}
\usepackage[lotdepth, lofdepth]{subfig}
\usepackage{ucs}
%\usepackage[utf8]{inputenc}

% \usepackage[final,expansion=true,protrusion=true,spacing=true,kerning=true]{microtype}
\usepackage{amsmath, epsfig}
\usepackage{amsfonts}
% \usepackage{amssymb}
\usepackage{algorithm}
\usepackage{algorithmic}
% \usepackage{easybmat}
\usepackage{footmisc}
\usepackage{bm}
% \usepackage{fullpage}
\renewcommand\algorithmiccomment[1]{// \textit{#1}}
\newcommand\independent{\protect\mathpalette{\protect\independenT}{\perp}}
\def\independenT#1#2{\mathrel{\rlap{$#1#2$}\mkern2mu{#1#2}}}
%

\title{Bayesian Hierarchical Models with Applications Survey Methods and Meta-analysis}
\author{Wei Wang \\ Oral Examination for the degree of \\ Master of Philosophy in
  the subject of \\ Statistics}
\date{Dec 2\textsuperscript{th}, 2013}

\begin{document}

\frame{\titlepage}

% \section[Outline]{}
% \frame{\tableofcontents}
%
% \section{Introduction}
% \subsection{Overview of Topics}
% 
% \section{Bayesian Analysis}
% \subsection{Single Parameter Model}

\begin{frame}
  \frametitle{Overview}
\tableofcontents
\end{frame}
\section{Hierarchical Non-parametric Models for Individual-level Meta-analysis}
\subsection{A Causal Framework for Meta-analysis}
\begin{frame}
  \begin{itemize}
  \item Meta-anayses synthesize evidence from multiple studies.
  \item Traditionally, meta-anlysis researchers go through thorough literature
    reviews and number extractions.
  \item Increasingly, researchers begin to have access to original data from a
    suite of studies and individual-level meta-analysis becomes feasible.
  \end{itemize}
\end{frame}

\begin{frame}
  \begin{itemize}
  \item Potential outcome framework is the standard tool for causal inference
    (Neyman, 1923; Rubin 1977).
  \item However, meta-analysis researchers rarely take on the potential outcome
    framework.
  \end{itemize}
\end{frame}

\begin{frame}
  \begin{itemize}
  \item Sobel et al. (2013) lays out an extended potential outcome framework for
    meta-analysis.
  \item The main goal in Sobel et al.(2013) is to codify different sources of
    heterogeneity that researchers tend to mesh together with the standard
    random-effect model.
  \end{itemize}
\end{frame}

\begin{frame}
  \frametitle{Recap of Sobel et al. (2013)}
  \begin{itemize}
  \item The extended potential outcome for meta-analysis is to consider all
    possible outcomes that individual $i$ can experience under every combination
    of study membership and treatment assignment,
    \begin{equation*}
      \label{eq:fpo}
      \bm Y_i=
      \begin{pmatrix}
        Y_i(1,1) & Y_i(1,2) & \cdots & Y_i(1,L)\\
        Y_i(2,1) & Y_i(2,2) & \cdots & Y_i(2,L)\\
        \vdots & \vdots & \ddots & \vdots\\
        Y_i(G,1) & Y_i(G,2) & \cdots & Y_i(G,L)
      \end{pmatrix}.
    \end{equation*}
  \end{itemize}
\end{frame}

\begin{frame}
  \frametitle{Recap of Sobel et al. (2013) Cont'd}
  \begin{itemize}
  \item[A1] Extended stable unit treatment value assumption: $Y_i(\bm s,\bm
    z)=Y_i(s_i, z_i)$ for all possible $\bm s$ and $\bm z$.
  \item[A2] Sampling assumption: $Y_i|s,z,x\sim F(y|s,z,x)$.
  \item[A3] Response consistency:
    $p(y(s,z)|S=s^{\prime\prime},X=x)=p(y(s,z)|S=s^{\prime\prime},X=x)$.
  \end{itemize}
\end{frame}

\begin{frame}
  \frametitle{Recap of Sobel et al. (2013) Cont'd}
  \begin{itemize}
  \item[A6] Strongly ignorable treatment assignment:\\
     \[\{Y(s,z):s = 1,...,G, z = 1,...,L\} \independent Z\mid S = s, X =
     x\] and
     \[0<P(Z=z\mid S=s, X=x)<1, \text{ for seen combinations of } s,z\]
   \item[A7] Ignorable study selection: \\
     \[\{Y(s,z): s = 1,...,G, z = 1,...,L\} \independent S \mid X\]
  \end{itemize}
\end{frame}

\begin{frame}
  \frametitle{Heterogeneity in Meta-analysis}
  \begin{itemize}
  \item There are several statistical issues in meta-analysis, including
    publication bias, choices of effect parameters and study qualities.
  \item The heterogeneity of effects across studies hinders research synthesis.
  \item Random-effects model (DerSimonion and Laird 1986) is the most popular
    choice, which state that effect sizes $\hat\theta_i$ from different studies
    have a prior distribution.
    \[\hat\theta_i=\theta_i+\varepsilon_i\text{  and  }\theta_i\sim \Psi(\cdot)\]
  \end{itemize}
\end{frame}

\begin{frame}
  \frametitle{Two Sources of Heterogeneity}
  \begin{itemize}
  \item Extended potential outcome framework provide insights into the possible
    sources of heterogeneity that could be introduced into meta-anlysis.
  \item First, the violation of the response consistency assumption (A3).
  \item Second, the violation of the no study selection assumption (A7).
  \end{itemize}
\end{frame}

\begin{frame}
  \frametitle{Two Sources of Heterogeneity Cont'd}
  \begin{itemize}
  \item These two sources are inherently different and it might not be meaningful
    to adopt a model that blends these two sources. 
  \item As common in causal inference, assessing how reasonable a particularly
    assumption is is very important for practical model choice and anlaysis.
  \item If (A7) is reasonable and (A3) is unlikely to hold, a random-effects
    model might be warranted.
  \item  If (A3) is reasonable and (A7) is unlikely to hold, random-effects
    models could be questionable.
  \end{itemize}
\end{frame}
\subsection{Non-parametric Models for Causal Inference}

\subsection{Future Work}

\section{Accurate Prediction Using Higly-biased Online Survey Data via MRP
  Correction}
\begin{frame}
  haha
\end{frame}
\subsection{MRP}
\subsection{Xbox Data}
\subsection{Results}
\subsection{Future Work}

\section{Sensitivity of Cross-validation in Hierarchical Model Comparison}
\begin{frame}
  haha
\end{frame}
\subsection{Future Work}
%%\bibliographystyle{elsa}
%%\bibliography{ref}

\end{document}
